\chapter{Metodologia de Pesquisa}
\label{cap:metodologia}

\section{Tipo e Objeto de Pesquisa}
\label{sec:tipo_e_Obj}

Informar o tipo da pesquisa: qualitativa, quantitativa ou qualiquantitativa. A pesquisa científica também pode ser classificada como: pesquisa bibliográfica, estudo de caso, pesquisa de campo, entre outras. Informar o objeto de pesquisa, o qual consiste no foco ou eixo central do estudo.

\section{População ou Amostra de Estudo}
\label{sec:amostra}

Uma população é um conjunto de pessoas, itens ou eventos. Nem sempre é conveniente ou possível examinar todos os membros de uma população. Assim, pode-se definir uma amostra, que é um subconjunto de uma população. Descreva a população ou amostra, e informe o número de participantes da pesquisa.

\subsection{Critérios de Inclusão}

Informar como serão compostos os grupos de participantes da pesquisa.

\subsection{Critérios de Exclusão}

Informar os critérios para a não inclusão de participantes informados no item anterior, e que não atendem aos propósitos da pesquisa.

\section{Descrição do Processo de Coleta de Dados}
\label{sec:coleta}

Informar o objetivo da aquisição dos dados, por exemplo para criação de uma base de dados, para criação ou validação do protótipo. Descrever detalhadamente a metodologia para aquisição dos dados, incluindo a forma de abordagem dos participantes, bem como o método de coleta de dados (execução de uma tarefa, questionário, entrevista etc.).

\subsection{Metodologia para revisão}

\section{Descrição do Processo de Análise de Dados}
\label{sec:analise}

Descrever o método matemático e estatístico (quantitativo e/ou qualitativo) a ser utilizado para a análise dos dados coletados.

\subsection{Análise Matemática e Estatística}

\subsection{Algoritmos e Frameworks}

Informar quais algoritmos ou \textit{frameworks} que serão utilizados na pesquisa. Eles podem ser necessários para realizar a aquisição de dados, a análise dos dados ou para a construção e validação do protótipo.

\section{Descrição do Protótipo}
\label{sec:prototipo}

\subsection{Visão Geral}

Este item deverá versar \underline{obrigatoriamente} sobre o Funcionamento do Sistema e o Interfaceamento entre as partes. Descreva a visão sobre “o que” seu dispositivo ou \textit{software} fará. A organização em subseções é livre ao autor.

Para descrever claramente a visão geral é útil a utilização de diagramas em blocos, inclusive para apresentar o interfaceamento entre as partes. 

\subsection{Lista de Funcionalidade e Atores}

Descrever em uma lista de funcionalidades que permita compreender como o sistema funcionará. Os atores do sistema deverão ser listados aqui também. Atores são todos aqueles que interagem com o sistema e que não fazem parte dele.

\subsection{Comunicação}
Como funciona toda a comunicação interna e externa do protótipo? Quais serão as tecnologias utilizadas para tal?

\subsection{Processamento}

Como será feito o processamento do protótipo? Quais serão as tecnologias utilizadas para tal? Descreva os requisitos mínimos necessários (HW e SW) para processamento das informações e ações.

\subsection{Interface Homem-Máquina}

Como será a interface do protótipo com o usuário? São utilizadas telas? Aqui devem ser mostrados protótipos das telas também.

\subsection{Sistemas Controlados Automaticamente}

O protótipo poderá processar e executar ações automaticamente? Como será feito esse processo? Existe alguma decisão que o sistema tomará sozinho?

\subsection{Aquisição de dados e Atuação}

Como e quando será feita alguma aquisição de dados? Qual será a forma de atuação do sistema em cada caso?
