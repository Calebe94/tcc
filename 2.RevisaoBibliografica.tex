\chapter{Revisão Bibliográfica}
\label{cap:revisao}

Os autores deverão, nesta seção, compor uma revisão teórica que explicite o contexto de inserção do problema levantado pelo seu projeto e dos temas principais para sua completa compreensão. 

Deve-se observar que assuntos técnicos devem ser trabalhos com cuidado para trazer informações pertinentes, relevantes e novas. Não serão aceitas revisões que descrevam o funcionamento básico de componentes eletrônicos, algoritmos ou \textit{frameworks}.

Neste sentido, pergunte-se, essa informação é relevante para compreender de forma completa o contexto no qual meu projeto está inserido? 

Exemplo -> \textit{Tema principal do trabalho:} Construção de um dispositivo capaz de identificar arritmias cardíacas. \textit{Exemplo de temas importantes a serem abordados na revisão:} o que são arritmias, quais são os tipos de arritmias cardíacas, como surgem, quais são suas consequências e formas de tratamento; efeito da doença a longo prazo, impactos sobre o sistema de saúde, dispositivos de prevenção, monitoramento e tratamento.

\textit{A organização das subseções é livre ao autor.}

\section{Trabalhos relacionados}
\label{sec:TrabalhosRelacionados}

Nesta seção, os autores deverão compor uma revisão sistemática a respeito de trabalhos relacionados ao apresentado pelo projeto.

Lembre-se de construir um comparativo através de pontos comuns entre os trabalhos e não simplesmente um resumo dos projetos analisados. 

\textit{A organização das subseções é livre ao autor.}
