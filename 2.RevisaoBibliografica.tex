\chapter{Revisão Bibliográfica}
\label{cap:revisao}

\section{Internet das Coisas}
\label{sec:InternetDasCoisas}

A Internet das Coisas (IoT) é uma tecnologia que tem ganhado cada vez mais destaque nos últimos anos. Ela se refere à conexão de dispositivos e objetos do cotidiano à internet, permitindo que eles troquem informações e sejam controlados remotamente. Com o aumento da quantidade de dispositivos conectados à internet, a quantidade de dados gerados também aumenta, o que pode sobrecarregar a rede e os dispositivos. Por isso, é importante utilizar técnicas de compressão de dados em dispositivos IoT com recursos limitados.

A IoT apresenta desafios, como as restrições dos objetos inteligentes, tais como processamento, memória e fonte de energia\cite{sampaio}. A conectividade entre aparelhos e o aprimoramento da internet têm permitido a inserção da IoT em quase todos os âmbitos da sociedade, facilitando o cotidiano do cidadão comum e contribuindo para o sucesso das organizações\cite{ricardo}. Além disso, a IoT tem sido aplicada em diversos contextos, como residências, transportes, saúde e indústria\cite{sampaio}.

Além disso, a transmissão de dados e amostras coletadas dos sensores em dispositivos IoT é uma tarefa fundamental, porém desafiadora, devido ao consumo energético e à necessidade de reduzir o uso do sistema de comunicação para melhorar o desempenho e a vida útil do sistema\cite{barros}.

\section{Desafios da IoT}
\label{sec:DesafiosDaIoT}

A Internet das Coisas (IoT) tem sido objeto de estudo devido aos desafios que apresenta. Um trabalho aborda os aspectos atuais da IoT, suas características e desafios, destacando a evolução da tecnologia e a interação de equipamentos com sensores e atuadores conectados à rede, tornando-os dispositivos inteligentes\cite{sol}. Além disso, a segurança na IoT é um tema crítico, pois a conectividade constante dos dispositivos a torna vulnerável a invasões. A necessidade de cuidados com a segurança da IoT é ressaltada, uma vez que os dispositivos IoT estão sempre ativos, o que os torna alvos potenciais para ataques cibernéticos\cite{ricardo}\cite{conexoes}. A heterogeneidade dos ambientes da IoT é outro desafio, demandando soluções para a interoperabilidade e integração dos diversos componentes\cite{scielo}. Além disso, o crescente volume de dados gerado pela IoT tem implicações para a transmissão e armazenamento, surgindo a necessidade de algoritmos de compressão de dados específicos para a IoT\cite{ricardo}.

\section{Compressão de Dados}
\label{sec:CompressãodeDados}

A compressão de dados é uma técnica amplamente utilizada para reduzir o tamanho dos dados sem comprometer a qualidade das informações. Ela desempenha um papel vital no contexto da IoT, especialmente em dispositivos com recursos limitados. Existem duas categorias principais de compressão de dados:

\subsection{Compressão de Dados Com Perdas}

A compressão de dados com perdas envolve a redução do tamanho dos dados, aceitando uma perda de qualidade ou precisão. Esta abordagem é frequentemente utilizada em aplicações onde a fidelidade exata dos dados não é essencial, como em transmissões de vídeo e áudio. Algoritmos como o JPEG para imagens, o MP3 para áudio e MPEG para vídeos são exemplos de técnicas de compressão com perdas\cite{sol}.

\subsection{Compressão de Dados Sem Perdas}

Por outro lado, a compressão de dados sem perdas visa reduzir o tamanho dos dados sem comprometer a integridade ou a qualidade dos mesmos. Esta abordagem é comumente utilizada em cenários onde a precisão dos dados é crucial, como em arquivos de texto e bancos de dados.
Algoritmos populares incluem o algoritmo Huffman, o Lempel-Ziv-Welch (LZW), o ZIP e GZIP. Essa técnica é adequada quando a integridade dos dados é fundamental\cite{ricardo}.

\section{Aplicação de Compressão de Dados na IoT}
\label{sec:AplicaçãodeCompressãodeDadosnaIoT}

A aplicação eficaz de técnicas de compressão de dados na IoT pode trazer benefícios significativos. A compressão de dados reduz a quantidade de dados a serem armazenados localmente nos dispositivos IoT e economiza largura de banda durante a transmissão. Isso é especialmente importante em cenários de IoT em que a comunicação ocorre por meio de redes com baixa taxa de transferência, como redes LPWAN (Low-Power Wide-Area Network)\cite{ricardo}.

\section{Desafios na Escolha de Técnicas de Compressão de Dados}
\label{sec:DesafiosnaEscolhadeTécnicasdeCompressãodeDados}

A compressão de dados em dispositivos IoT apresenta desafios específicos, como a necessidade de equilibrar a taxa de compressão com o consumo de energia e a capacidade de processamento limitada desses dispositivos. Estudos recentes têm explorado algoritmos de compressão de dados específicos para dispositivos IoT, visando otimizar a transmissão e o armazenamento de dados nesse contexto\cite{ricardo}.

A compreensão das técnicas de compressão de dados, tanto com perdas quanto sem perdas, é essencial para a eficiência na transmissão e armazenamento de informações, especialmente em ambientes com recursos limitados, como os dispositivos IoT.

\section{Trabalhos relacionados}
\label{sec:TrabalhosRelacionados}

Para realizar a revisão bibliográfica deste TCC, foram utilizadas diversas fontes de informação. Um dos trabalhos encontrados apresenta uma revisão sistemática da literatura sobre a IoT e seus desafios, incluindo a compressão de dados, realizada em artigos, monografias e dissertações\cite{fachini}.

Outro trabalho encontrado apresenta uma revisão da literatura sobre a compressão de dados com perdas e sem perdas, incluindo algoritmos de compressão utilizados em dispositivos IoT\cite{silvarevisao}. Além disso, foram encontrados artigos que abordam técnicas específicas de compressão de dados, como a compressão de imagens e vídeos\cite{almeida}\cite{silvacompressaovideo}.

Ao comparar os trabalhos encontrados, é possível perceber que a compressão de dados é uma técnica importante para a IoT, uma vez que os dispositivos IoT possuem recursos limitados.
Além disso, a compressão de dados com perdas é mais utilizada em imagens e vídeos, enquanto a compressão sem perdas é mais utilizada em arquivos de texto e bancos de dados. Diversos algoritmos de compressão são utilizados em dispositivos IoT, como o algoritmo LZ77, o algoritmo Huffman e o algoritmo LZW.

Em resumo, a compressão de dados é uma técnica importante para a IoT, uma vez que os dispositivos IoT possuem recursos limitados. Existem duas técnicas principais de compressão de dados: a compressão com perdas e a compressão sem perdas. Diversos algoritmos de compressão são utilizados em dispositivos IoT, e a escolha do algoritmo depende do tipo de dado a ser comprimido.

%\textit{A organização das subseções é livre ao autor.}
