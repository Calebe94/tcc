\chapter{Especificação Técnica}

\section{Análise de Contexto}

\subsection{Visão Geral}

O sistema é uma ferramenta de compressão e descompressão de dados, utilizando os algoritmos Huffman, Deflate, RLE e LZW. A estrutura do projeto é organizada em diferentes diretórios: \texttt{src} contém o código-fonte principal, \texttt{lib} abriga bibliotecas de terceiros e implementações adicionais, enquanto \texttt{bin} é o diretório de saída para arquivos objeto e o executável final.

\subsection{Condições Restritivas}

\subsubsection{Custos}

Não há restrições específicas de custo para o desenvolvimento do protótipo. Porém, visando a aplicação em dispositivos com recursos limitados, será considerada a eficiência de uso de recursos computacionais.

\subsubsection{Físicas e Ambientais}

Não há restrições físicas ou ambientais impostas ao funcionamento do protótipo. A aplicação não requer características específicas em termos de ambiente operacional.

\subsubsection{Tecnológicas}

O acesso à tecnologia não é restrito, porém, a aplicação foi desenvolvida considerando dispositivos com recursos limitados.

\subsubsection{Energização}

Não há requisitos especiais de energia para o funcionamento do software.

\subsubsection{Interferências devido ao meio}

Não foram identificadas interferências externas que restrinjam o funcionamento do protótipo.

\subsection{Benefícios e Impactos}

\subsubsection{Econômicos}

A implementação do protótipo visa otimizar o uso de recursos de armazenamento e transmissão de dados, potencialmente resultando em economia de recursos de hardware e largura de banda.

\subsubsection{Operacionais}

O protótipo foi desenvolvido para facilitar a compressão e descompressão de dados, não impactando significativamente a rotina operacional do ambiente.

\subsubsection{Estratégicos}

A utilização do protótipo está alinhada com estratégias de otimização de recursos em dispositivos com capacidades limitadas.

\subsubsection{Políticos}

Não há impactos políticos previstos com a implementação do protótipo.

\subsubsection{Sociais}

A aplicação pode contribuir positivamente ao permitir o uso eficiente de recursos em dispositivos IoT, beneficiando a comunidade de desenvolvimento de IoT e usuários finais.

\section{Análise Funcional e de Requisitos Tecnológicos}

\subsection{Lista de Funcionalidades e Atores}

\begin{itemize}
    \item Funcionalidades:
        \begin{itemize}
            \item Compressão e descompressão de dados.
            \item Suporte aos algoritmos Huffman, Deflate, RLE e LZW.
        \end{itemize}
    \item Atores:
        \begin{itemize}
            \item Usuários finais.
        \end{itemize}
\end{itemize}

\subsection{Comunicação}

A comunicação interna do protótipo é direcionada à execução dos algoritmos de compressão e descompressão, enquanto a comunicação externa é realizada através da interface de linha de comando.

\subsection{Processamento}

O processamento é realizado por meio de algoritmos de compressão implementados em C. Os requisitos mínimos de hardware incluem um processador básico capaz de executar operações de compressão e descompressão de dados.

\subsection{Interface Homem-Máquina}

O protótipo possui uma interface de linha de comando, permitindo a interação com o usuário para seleção e execução dos algoritmos de compressão e descompressão.

\subsection{Sistemas Controlados Automaticamente}

O protótipo não executa automaticamente, requerendo a entrada do usuário para a seleção e execução das operações de compressão e descompressão.

\section{Análise da Arquitetura do Sistema}

\subsection{Hardware}

O software não tem dependência de hardware específico, sendo executável em dispositivos comuns.

\subsection{Software}

O funcionamento do software depende do sistema operacional que suporta a linguagem C, juntamente com as bibliotecas específicas utilizadas para os algoritmos de compressão.
