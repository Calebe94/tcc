\chapter{Especificação Técnica}
\label{cap:especificacao}

\section{Análise de Contexto}
\label{sec:analiseContexto}

\subsection{Visão Geral}

Este item deverá versar \underline{obrigatoriamente} sobre o Funcionamento do Sistema e o Interfaceamento entre as partes. Descreva a visão sobre “o que” seu dispositivo ou \textit{software} fará. A organização em subseções é livre ao autor.

Para descrever claramente a visão geral é útil a utilização de diagramas em blocos, inclusive para apresentar o interfaceamento entre as partes. 

\subsection{Condições Restritivas}

Este item deverá versar \underline{obrigatoriamente}, de acordo com a aplicação no trabalho, sobre as seguintes condições restritivas

\subsubsection{Custos}

Versar sobre a existência de alguma restrição de custo. O protótipo será de baixo custo? Existe um limite de orçamento? Em uma visão de futuro, seu produto deve seguir alguma restrição de custo de operação ou de aquisição?

\subsubsection{Físicas e Ambientais}

Existe alguma restrição de funcionamento do protótipo considerando interferências físicas ou ambientais? Sua composição deve atender requisitos mínimos exigidos por alguma característica física ou ambiental de sua aplicação?

\subsubsection{Tecnológicas}

O acesso à tecnologia, ou dificuldade de acesso, pode restringir sua utilização no projeto? Existem aspectos de projeto que exigem ou impeçam a utilização de alguma tecnologia específica?

\subsubsection{Energização}
Existe alguma condição de energização do protótipo que restringe o uso de alguma tecnologia? 

\subsubsection{Interferências devido ao meio}

Existe alguma interferência devido ao meio que trará alguma restrição ao funcionamento do protótipo?

\subsection{Benefícios e Impactos}

Este item deverá versar \underline{obrigatoriamente}, de acordo com a aplicação no trabalho, sobre os seguintes benefícios e impactos

\subsubsection{Econômicos}

A implantação gerará custos ou lucros? Seu protótipo gerará economia de insumos ou outros benefícios?

\subsubsection{Operacionais}

Quais serão os prós e contras da operacionalização do projeto? Será necessário mudar a rotina do ambiente em que seu protótipo será aplicado? As mudanças impactarão positiva ou negativamente?

\subsubsection{Estratégicos}

A implantação do seu protótipo exigirá estratégias de curto ou longo prazo? A implantação do seu protótipo faz parte da estratégia de solução do problema e está relacionada a alguma outra ação? 

\subsubsection{Políticos}

Será necessário compor novas políticas públicas para implantação do seu protótipo? Seu protótipo poderá promover mudanças de aspecto legal em algum setor?

\subsubsection{Sociais}
Quais serão os benefícios ou impactos sentidos pela população ou meio-ambiente a partir da implementação do seu protótipo?

\section{Análise Funcional e de Requisitos Tecnológicos}
\label{sec:analisefuncional}

A análise funcional descreve aspectos relacionados ao funcionamento do sistema. A análise de requisitos tecnológicos nomeia as tecnologias que serão utilizadas para a construção do protótipo. Aqui também podem ser utilizadas diagramas em blocos para ilustrar a descrição (que deve ser completa).

\subsection{Lista de Funcionalidade e Atores}

Descrever em uma lista de funcionalidades que permita compreender como o sistema funcionará. Os atores do sistema deverão ser listados aqui também. Atores são todos aqueles que interagem com o sistema e que não fazem parte dele.

\subsection{Comunicação}
Como funciona toda a comunicação interna e externa do protótipo? Quais serão as tecnologias utilizadas para tal?

\subsection{Processamento}

Como será feito o processamento do protótipo? Quais serão as tecnologias utilizadas para tal? Descreva os requisitos mínimos necessários (HW e SW) para processamento das informações e ações.

\subsection{Interface Homem-Máquina}

Como será a interface do protótipo com o usuário? São utilizadas telas? Aqui devem ser mostrados protótipos das telas também.

\subsection{Sistemas Controlados Automaticamente}

O protótipo poderá processar e executar ações automaticamente? Como será feito esse processo? Existe alguma decisão que o sistema tomará sozinho?

\subsection{Aquisição de dados e Atuação}

Como e quando será feita alguma aquisição de dados? Qual será a forma de atuação do sistema em cada caso?

\section{Análise da Arquitetura do Sistema}

\subsection{Hardware}

Este item deverá versar sobre o funcionamento do dispositivo de \textit{hardware} e suas circuitarias. \underline{obrigatoriamente} os autores devem incorporar ao texto Diagramas de Blocos para a descrição do funcionamento do \textit{hardware} e a interconexão de dispositivos, bem como Diagramas de Fluxo para a descrição lógica do \textit{firmware}.

\subsection{Software}

Este item deverá versar sobre o funcionamento do \textit{software} e suas dependências. obrigatoriamente os autores deverão incorporar ao texto, por meio de seções específicas, Modelo Entidade Relacionamento (MER), Diagrama Entidade Relacionamento (DER), Use Cases, Diagramas de Sequência e Diagramas de Classe.
