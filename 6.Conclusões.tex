\chapter{Conclusões}
\label{cap:conclusoes}

Neste trabalho, apresentamos uma ferramenta de compressão de dados utilizando diferentes algoritmos e realizamos testes em ambientes distintos para avaliar seu desempenho. Com base nos resultados obtidos, as seguintes conclusões podem ser destacadas:

\section{Análise dos Resultados}

Os testes realizados demonstraram a influência significativa do hardware no desempenho da ferramenta. O ambiente x86 apresentou tempos de processamento consideravelmente menores em comparação ao processador ARMv6, evidenciando a importância do hardware na eficiência dos algoritmos de compressão.

\section{Considerações Finais}

Com base nos dados coletados, é possível inferir que a escolha do hardware é crucial para a otimização do desempenho da ferramenta. A análise dos resultados revelou diferenças marcantes nos tempos de compressão e descompressão entre os ambientes testados, indicando que investimentos em hardware mais robusto podem resultar em melhorias significativas no desempenho da ferramenta de compressão de dados.

\section{Trabalhos Futuros}

Como possíveis trabalhos futuros, sugerimos:
\begin{itemize}
    \item Explorar e implementar estratégias de otimização de algoritmos para melhorar o desempenho em ambientes de hardware menos robusto.
    \item Realizar testes em uma variedade maior de hardware para avaliar a escalabilidade e generalização do desempenho da ferramenta.
    \item Investigar a aplicação de novos algoritmos de compressão e comparar seu desempenho com os algoritmos atualmente implementados.
    \item Desenvolver versões específicas da ferramenta para diferentes tipos de hardware, visando a otimização e adaptação para cada ambiente específico.
\end{itemize}

Essas possibilidades de trabalhos futuros visam aprimorar a ferramenta, explorando diferentes estratégias e algoritmos para melhorar sua eficiência e aplicabilidade em uma variedade de ambientes de hardware.
