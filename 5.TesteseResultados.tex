\chapter{Teste e Resultados}
\label{cap:teste-resultados}

Neste capítulo, descrevemos os testes de validação realizados para a ferramenta de compressão de dados utilizando diferentes algoritmos. Estes testes foram conduzidos em dois diferentes ambientes: um processador ARMv6 com 256MB de memória RAM e um processador x86 (Ryzen 6) com 32GB de RAM. Os resultados obtidos são apresentados a seguir:

\section{Testes no Processador ARMv6}

Os testes realizados no processador ARMv6 apresentaram os seguintes resultados, conforme mostrado na Tabela \ref{tab:exemplo}.

\begin{table}[htbp]
  \centering
  \caption{Resultados obtidos no processador ARMv6}
  \begin{tabular}{|c|c|c|c|}
  \hline
  Algoritmo & Operação & Tempo (s) & Tamanho do arquivo (kB) \\
  \hline
  Huffman & Compressão & 0.669 & 176 \\
  Huffman & Descompressão & 0.098 & 688 \\
  Deflate & Compressão & 0.699 & 176 \\
  Deflate & Descompressão & 0.085 & 688 \\
  RLE & Compressão & 0.806 & 1232 \\
  RLE & Descompressão & 0.086 & 680 \\
  LZW & Compressão & 40.834 & 228 \\
  LZW & Descompressão & 0.146 & 688 \\
  \hline
  \end{tabular}
  \label{tab:exemplo}
\end{table}

Os resultados obtidos no processador ARMv6 indicam tempos de processamento mais prolongados em comparação com o ambiente x86, especialmente para o algoritmo LZW.

\section{Testes no Processador x86}

Já os testes no processador x86, com um ambiente mais robusto em termos de hardware, apresentaram os seguintes resultados, conforme mostrado na Tabela \ref{tab:x86}.

\begin{table}[htbp]
  \centering
  \caption{Resultados obtidos no processador x86}
  \begin{tabular}{|c|c|c|c|}
  \hline
  Algoritmo & Operação & Tempo (s) & Tamanho do arquivo (kB) \\
  \hline
  Huffman & Compressão & 0.161 & 176 \\
  Huffman & Descompressão & 0.008 & 688 \\
  Deflate & Compressão & 0.161 & 176 \\
  Deflate & Descompressão & 0.008 & 688 \\
  RLE & Compressão & 0.024 & 1232 \\
  RLE & Descompressão & 0.005 & 680 \\
  LZW & Compressão & 3.515 & 228 \\
  LZW & Descompressão & 0.011 & 688 \\
  \hline
  \end{tabular}
  \label{tab:x86}
\end{table}

No ambiente x86, os tempos de processamento foram significativamente menores em comparação com o ARMv6, indicando uma melhoria considerável de desempenho, especialmente para o algoritmo LZW.

\section{Discussão dos Resultados}

Os resultados dos testes demonstram claramente a influência do hardware no desempenho da ferramenta de compressão de dados. Ambos os ambientes apresentaram variações significativas nos tempos de compressão e descompressão, destacando a importância do hardware na eficiência dos algoritmos utilizados.

Além disso, os dados obtidos fornecem informações valiosas sobre os tempos de execução dos diferentes algoritmos em diferentes ambientes, contribuindo para a compreensão do desempenho da ferramenta em contextos variados.
