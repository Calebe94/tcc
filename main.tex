%  -----------------------------------------------------------------------
% |                Modelo de documento Latex para TCC do                  |
% |     curso de Engenharia da Computação da Universidade Positivo        |
%  -----------------------------------------------------------------------
% |     Produção: Eduardo J Alberti      Revisão: Veronica I. Quandt      |
%  -----------------------------------------------------------------------
% |                             Versão 2.0                                |
%  -----------------------------------------------------------------------

\documentclass{TCC_UP}

%  -----------------------------------------------------------------------
% |                 Informações para construção da Capa                   |
% |           Título, Autores, Orientador, Universidade e Ano             |
%  -----------------------------------------------------------------------

\titulo{Modelo de Documento \\de Trabalho de Conclusão de Curso}
    \tipotrabalho{Monografia }
    %\TipoPesquisa % Descomentar essa linha para projetos de Pesquisa

\autor{Nome do Autor 1 \\ Nome do Autor 2}
    \curso{Engenharia da Computação }
    \escola{Escola Politécnica }

\orientador{Nome do Orientador}
%\coorientador{Nome do Coorientador} % Descomentar se necessário

\makeatletter
\hypersetup{pdftitle={\@title}, pdfauthor={\@author},
            pdfsubject={\imprimirpreambulo},
            pdfkeywords={\imprimircurso}{\imprimirinstituicao}{TCC}, colorlinks=false,bookmarksdepth=4}
\makeatother
\makeindex

\begin{document}
    \pretextual
    \selectlanguage{brazil}
    \frenchspacing

%  -----------------------------------------------------------------------
% |                 Imprime a Capa e a Folha de Rosto                     |
%  -----------------------------------------------------------------------
    \imprimircapa
    \imprimirfolhaderosto

%  -----------------------------------------------------------------------
% |                       Constrói a Errata                               |
%  -----------------------------------------------------------------------
% | A errata é um elemento opcional que deve ser usado apenas quando não  |
% |   houver mais possibilidade de correção de determinado fragmento de   |
% |  texto. Substitua os elementos abaixo, e descomente, caso necessário  |
%  -----------------------------------------------------------------------
%    \begin{errata}
%        FERRIGNO, C. R. A. \textbf{Tratamento de neoplasias ósseas apendiculares com reimplantação de enxerto ósseo autólogo autoclavado associado ao plasma rico em plaquetas}: estudo crítico na cirurgia de preservação de membro em cães. 2011. 128 f. Tese (Livre-Docência) - Faculdade de Medicina Veterinária e Zootecnia, Universidade de São Paulo, São Paulo, 2011.
%            \begin{table}[htb]
%                \center
%                \begin{tabular}{|p{1.4cm}|p{1cm}|p{3cm}|p{3cm}|}
%                    \hline
%                    \textbf{Folha} & \textbf{Linha} & \textbf{Onde se lê} &
%                    \textbf{Leia-se}\\
%                    \hline
%                    1 & 10 & auto-conclavo & autoconclavo\\
%                    \hline
%                \end{tabular}
%        \end{table}
%    \end{errata}

%  -----------------------------------------------------------------------
% |                          Folha de Aprovação                           |
%  -----------------------------------------------------------------------
% | Na versão final do trabalho, a ser entregue correção de considerações |
% |  da banca de avaliação, será necessário incluir a Folha de Aprovação. |
% |   A Folha de Aprovação será cedida pela coordenação de TCC e poderá   |
% |               ser incluída por meio do comando abaixo.                |
%  -----------------------------------------------------------------------
    % \includepdf{Folha_de_Aprovacao.pdf}

%  -----------------------------------------------------------------------
% |                        Elementos Opcionais                            |
%  -----------------------------------------------------------------------
%  ------------------------- Dedicatória ---------------------------------
%    \begin{dedicatoria}
%       \vspace*{\fill}
%       \begin{adjustwidth}{9cm}{0cm}
%            Tem a finalidade de prestar homenagem a alguém e é opcional.
%       \end{adjustwidth}
%    \end{dedicatoria}
%  ----------------------- Agradecimentos --------------------------------
%    \begin{agradecimentos}
%        Os agradecimentos são opcionais e mencionam as pessoas e/ou
%        instituições que contribuíram para o desenvolvimento do trabalho.
%
%        Separe os agradecimentos em parágrafos quando o texto for longo.
%    \end{agradecimentos}
%  ------------------------- Epígrafe -----------------------------------
%    \begin{epigrafe}
%        \vspace*{\fill}
%    	\begin{adjustwidth}{9cm}{0cm}
%    		\textit{Utilizar a parte inferior da página, como neste exemplo.
%           Texto justificado e alinhado pela parte inferior da página. Em geral, contém um fragmento de texto (ex. citação curta, composição poética etc.). É relacionado ao tema ou à motivação do trabalho, e é opcional.}
%    	\end{adjustwidth}
%    \end{epigrafe}

%  -----------------------------------------------------------------------
% |                       Resumo - Obrigatório                            |
%  -----------------------------------------------------------------------
    \setlength{\absparsep}{18pt}
    \begin{resumo}
        Aqui deve-se descrever um resumo do trabalho abrangendo-o em sua totalidade. É proibido o uso de citações ou inserção de figuras, tabelas e outros elementos. Deve-se, ainda, descrever um conjunto de 3 a 5 palavras-chave separadas por um ponto.

        \textbf{Palavras-chave}: palavra-chave 1. palavra-chave 2. palavra-chave 3.
    \end{resumo}

%  -----------------------------------------------------------------------
% |                       Abstract - Obrigatório                          |
%  -----------------------------------------------------------------------
    \begin{resumo}[Abstract]
     \begin{otherlanguage*}{english}
       This is the english abstract.

       \textbf{Keywords}: latex. abntex. text editoration.
     \end{otherlanguage*}
    \end{resumo}

%  -----------------------------------------------------------------------
% |                         Listas Opcionais                              |
%  -----------------------------------------------------------------------
%  ------------------------ Lista de Figuras -----------------------------
    \pdfbookmark[0]{\listfigurename}{lof}
    \listoffigures*
    \cleardoublepage

%  ----------------------- Lista de Gráficos -----------------------------
    \pdfbookmark[0]{\listofgraficosname}{los}
    \listofgraficos*
    \cleardoublepage

%  ------------------------ Lista de Quadros -----------------------------
    \pdfbookmark[0]{\listofquadrosname}{loq}
    \listofquadros*
    \cleardoublepage

%  ------------------------ Lista de Tabelas -----------------------------
    \pdfbookmark[0]{\listtablename}{lot}
    \listoftables*
    \cleardoublepage

%  ------------------ Lista de Abreviaturas e Siglas ---------------------
%  ------------ Utilize \nomenclature{Sigla}{Definição} ------------------
    \pdfbookmark[0]{\nomname}{las}
    \printnomenclature
    \cleardoublepage

%  ------------------------ Lista de Símbolos ----------------------------
%  ----------- Note que esta lista deve ser criada manualmente -----------
    \begin{simbolos}
      \item[$ \Omega $] Ohm
      \item[$ \Delta V $] Variação de tensão
    \end{simbolos}

%  -----------------------------------------------------------------------
% |                           Cria Sumário                                |
%  -----------------------------------------------------------------------
    \tableofcontents*
    \cleardoublepage

%  -----------------------------------------------------------------------
% |                   Inclui os arquivos de capítulos                     |
%  -----------------------------------------------------------------------
    \textual
    \pagestyle{simple}

% Inclusão de capítulos. Altera componentes de acordo com Tipo do Trabalho
    \chapter{Introdução}
\label{cap:introducao}

A Internet das Coisas (IoT) tem revolucionado a forma como interagimos com o ambiente ao nosso redor, conectando dispositivos, sensores e sistemas a uma rede global, permitindo a coleta e troca de informações em tempo real. Esta evolução tecnológica tem encontrado aplicações em diversos setores, desde a automação residencial até a otimização de processos industriais e sistemas de saúde avançados. No entanto, à medida que a IoT se expande, torna-se cada vez mais evidente a necessidade de enfrentar os desafios associados ao armazenamento e transmissão de dados em dispositivos com recursos limitados.

\section{Problema}
\label{sec:problema}

O advento da Internet das Coisas (IoT) revolucionou a forma como interagimos com o mundo ao nosso redor, conectando dispositivos, sensores e sistemas a uma rede global. No entanto, esse crescimento exponencial de dispositivos IoT trouxe consigo desafios significativos relacionados ao armazenamento e transmissão eficiente de dados. Muitos dispositivos IoT são caracterizados por recursos limitados, incluindo capacidade de armazenamento, capacidade de processamento e largura de banda de transmissão. Este cenário levanta a seguinte questão fundamental: como gerenciar eficazmente os dados gerados por dispositivos IoT em um ambiente de recursos limitados?

\section{Justificativa}
\label{sec:justificativa}

A justificativa para a realização deste trabalho reside na necessidade premente de abordar os desafios inerentes à IoT, particularmente no que diz respeito ao gerenciamento eficiente de dados em dispositivos com recursos limitados. Dispositivos IoT são amplamente utilizados em diversas aplicações, incluindo saúde, agricultura, manufatura e cidades inteligentes, e o volume de dados gerados por esses dispositivos continua a crescer exponencialmente. A compressão de dados surge como uma estratégia crucial para otimizar a utilização de recursos, economizar energia e melhorar a eficiência na transmissão de dados em redes com baixa taxa de transferência. Portanto, este trabalho se justifica pela necessidade de explorar e avaliar as técnicas de compressão de dados mais apropriadas para dispositivos IoT com recursos limitados, contribuindo para a resolução deste desafio tecnológico.

\section{Objetivo Geral}
\label{sec:ObjGeral}

O objetivo geral deste trabalho é investigar e analisar as técnicas de compressão de dados mais adequadas para dispositivos IoT com recursos limitados de armazenamento e transmissão. Isso envolve a compreensão das diferentes abordagens de compressão disponíveis, suas vantagens, desvantagens e áreas de aplicação ideais em contextos de IoT. Além disso, o objetivo geral inclui a realização de estudos de caso práticos para avaliar a eficácia dessas técnicas em cenários reais de implementação de dispositivos IoT.

\section{Objetivos Específicos}
\label{sec:ObjEspecificos}

Para atingir o objetivo geral, os seguintes objetivos específicos foram estabelecidos:

\begin{itemize}
\item Realizar uma revisão abrangente da literatura existente sobre técnicas de compressão de dados, com foco nas estratégias desenvolvidas para dispositivos IoT com recursos limitados.
\item Comparar as técnicas de compressão de dados sem perdas e com perdas, identificando suas características, vantagens e limitações.
\item Implementar um estudo de caso prático que envolva a coleta de dados por dispositivos IoT em um ambiente controlado.
\item Aplicar diversas técnicas de compressão de dados aos dados coletados e analisar os resultados em termos de economia de recursos, qualidade dos dados e eficiência da transmissão.
\end{itemize}

    \chapter{Revisão Bibliográfica}
\label{cap:revisao}

Os autores deverão, nesta seção, compor uma revisão teórica que explicite o contexto de inserção do problema levantado pelo seu projeto e dos temas principais para sua completa compreensão. 

Deve-se observar que assuntos técnicos devem ser trabalhos com cuidado para trazer informações pertinentes, relevantes e novas. Não serão aceitas revisões que descrevam o funcionamento básico de componentes eletrônicos, algoritmos ou \textit{frameworks}.

Neste sentido, pergunte-se, essa informação é relevante para compreender de forma completa o contexto no qual meu projeto está inserido? 

Exemplo -> \textit{Tema principal do trabalho:} Construção de um dispositivo capaz de identificar arritmias cardíacas. \textit{Exemplo de temas importantes a serem abordados na revisão:} o que são arritmias, quais são os tipos de arritmias cardíacas, como surgem, quais são suas consequências e formas de tratamento; efeito da doença a longo prazo, impactos sobre o sistema de saúde, dispositivos de prevenção, monitoramento e tratamento.

\textit{A organização das subseções é livre ao autor.}

\section{Trabalhos relacionados}
\label{sec:TrabalhosRelacionados}

Nesta seção, os autores deverão compor uma revisão sistemática a respeito de trabalhos relacionados ao apresentado pelo projeto.

Lembre-se de construir um comparativo através de pontos comuns entre os trabalhos e não simplesmente um resumo dos projetos analisados. 

\textit{A organização das subseções é livre ao autor.}

    \ifx\Pesquisa\undefined
        \chapter{Estudo de Caso}
\label{cap:estudo}

\section{O Projeto}
A criação de uma ferramenta de compressão de dados em linguagem C foi um processo desafiador e empolgante. Inicialmente, o foco foi entender a lógica e os fundamentos por trás dos algoritmos de compressão mais comuns, como Huffman, Deflate, RLE e LZW.

O desenvolvimento teve início com a compreensão aprofundada de cada algoritmo, suas vantagens, desvantagens e as situações ideais para sua aplicação. Isso incluiu uma exploração detalhada das estruturas de dados e dos métodos de manipulação de informações, fundamentais para a eficiência de compressão.

\section{Implementação}
A implementação dos algoritmos foi um passo crucial. Cada um deles exigiu um entendimento profundo de suas especificidades e da lógica subjacente. Isso envolveu a tradução desses conceitos em código C, garantindo ao mesmo tempo eficiência e robustez.

Durante o desenvolvimento, a modularidade e a eficiência do código foram priorizadas. Isso implicou na criação de funções bem definidas para cada parte do processo de compressão e descompressão, mantendo um código limpo e de fácil compreensão.

\section{Compilação e Execução}
Após a implementação, foi essencial realizar testes abrangentes para validar a funcionalidade dos algoritmos. O processo de compilação e execução foi feito em diferentes plataformas para garantir a portabilidade e o correto funcionamento da ferramenta em ambientes diversos.

A execução dos testes de compressão e descompressão em diferentes tipos de dados foi crucial para verificar a eficácia e a robustez dos algoritmos implementados.

Além disso, a ferramenta foi submetida a testes de desempenho e otimização. O objetivo foi melhorar a eficiência dos algoritmos e garantir um desempenho ideal em situações diversas.

\section{Considerações Finais}
O desenvolvimento dessa ferramenta proporcionou uma experiência enriquecedora. Foi um processo que exigiu não apenas conhecimento técnico, mas também habilidades de otimização, resolução de problemas e compreensão aprofundada dos algoritmos de compressão de dados.

Ao final do processo, foi gratificante ter uma ferramenta capaz de comprimir e descomprimir dados com eficiência, utilizando uma gama de algoritmos reconhecidos e testados.

    \else
        \chapter{Metodologia de Pesquisa}
\label{cap:metodologia}

\section{Tipo e Objeto de Pesquisa}
\label{sec:tipo_e_Obj}

Informar o tipo da pesquisa: qualitativa, quantitativa ou qualiquantitativa. A pesquisa científica também pode ser classificada como: pesquisa bibliográfica, estudo de caso, pesquisa de campo, entre outras. Informar o objeto de pesquisa, o qual consiste no foco ou eixo central do estudo.

\section{População ou Amostra de Estudo}
\label{sec:amostra}

Uma população é um conjunto de pessoas, itens ou eventos. Nem sempre é conveniente ou possível examinar todos os membros de uma população. Assim, pode-se definir uma amostra, que é um subconjunto de uma população. Descreva a população ou amostra, e informe o número de participantes da pesquisa.

\subsection{Critérios de Inclusão}

Informar como serão compostos os grupos de participantes da pesquisa.

\subsection{Critérios de Exclusão}

Informar os critérios para a não inclusão de participantes informados no item anterior, e que não atendem aos propósitos da pesquisa.

\section{Descrição do Processo de Coleta de Dados}
\label{sec:coleta}

Informar o objetivo da aquisição dos dados, por exemplo para criação de uma base de dados, para criação ou validação do protótipo. Descrever detalhadamente a metodologia para aquisição dos dados, incluindo a forma de abordagem dos participantes, bem como o método de coleta de dados (execução de uma tarefa, questionário, entrevista etc.).

\subsection{Metodologia para revisão}

\section{Descrição do Processo de Análise de Dados}
\label{sec:analise}

Descrever o método matemático e estatístico (quantitativo e/ou qualitativo) a ser utilizado para a análise dos dados coletados.

\subsection{Análise Matemática e Estatística}

\subsection{Algoritmos e Frameworks}

Informar quais algoritmos ou \textit{frameworks} que serão utilizados na pesquisa. Eles podem ser necessários para realizar a aquisição de dados, a análise dos dados ou para a construção e validação do protótipo.

\section{Descrição do Protótipo}
\label{sec:prototipo}

\subsection{Visão Geral}

Este item deverá versar \underline{obrigatoriamente} sobre o Funcionamento do Sistema e o Interfaceamento entre as partes. Descreva a visão sobre “o que” seu dispositivo ou \textit{software} fará. A organização em subseções é livre ao autor.

Para descrever claramente a visão geral é útil a utilização de diagramas em blocos, inclusive para apresentar o interfaceamento entre as partes. 

\subsection{Lista de Funcionalidade e Atores}

Descrever em uma lista de funcionalidades que permita compreender como o sistema funcionará. Os atores do sistema deverão ser listados aqui também. Atores são todos aqueles que interagem com o sistema e que não fazem parte dele.

\subsection{Comunicação}
Como funciona toda a comunicação interna e externa do protótipo? Quais serão as tecnologias utilizadas para tal?

\subsection{Processamento}

Como será feito o processamento do protótipo? Quais serão as tecnologias utilizadas para tal? Descreva os requisitos mínimos necessários (HW e SW) para processamento das informações e ações.

\subsection{Interface Homem-Máquina}

Como será a interface do protótipo com o usuário? São utilizadas telas? Aqui devem ser mostrados protótipos das telas também.

\subsection{Sistemas Controlados Automaticamente}

O protótipo poderá processar e executar ações automaticamente? Como será feito esse processo? Existe alguma decisão que o sistema tomará sozinho?

\subsection{Aquisição de dados e Atuação}

Como e quando será feita alguma aquisição de dados? Qual será a forma de atuação do sistema em cada caso?

    \fi
    \chapter{Desenvolvimento}
\label{cap:desenvolvimento}

Neste capítulo os autores deverão compor uma descrição a respeito do processo de desenvolvimento do protótipo, abordando nuances de projeto e como o desenvolvimento atendeu à requisitos descritos no capítulo anterior. A organização em subseções é livre aos autores.
    \chapter{Teste e Resultados}
\label{cap:teste-resultados}

Neste capítulo, descrevemos os testes de validação realizados para a ferramenta de compressão de dados utilizando diferentes algoritmos. Estes testes foram conduzidos em dois diferentes ambientes: um processador ARMv6 com 256MB de memória RAM e um processador x86 (Ryzen 6) com 32GB de RAM. Os resultados obtidos são apresentados a seguir:

\section{Testes no Processador ARMv6}

Os testes realizados no processador ARMv6 apresentaram os seguintes resultados:

\begin{center}
\begin{tabular}{|c|c|c|c|}
\hline
Algoritmo & Operação & Tempo (s) & Tamanho do arquivo (kB) \\
\hline
Huffman & Compressão & 0.669 & 176 \\
Huffman & Descompressão & 0.098 & 688 \\
Deflate & Compressão & 0.699 & 176 \\
Deflate & Descompressão & 0.085 & 688 \\
RLE & Compressão & 0.806 & 1232 \\
RLE & Descompressão & 0.086 & 680 \\
LZW & Compressão & 40.834 & 228 \\
LZW & Descompressão & 0.146 & 688 \\
\hline
\end{tabular}
\end{center}

Os resultados obtidos no processador ARMv6 indicam tempos de processamento mais prolongados em comparação com o ambiente x86, especialmente para o algoritmo LZW.

\section{Testes no Processador x86}

Já os testes no processador x86, com um ambiente mais robusto em termos de hardware, apresentaram os seguintes resultados:

\begin{center}
\begin{tabular}{|c|c|c|c|}
\hline
Algoritmo & Operação & Tempo (s) & Tamanho do arquivo (kB) \\
\hline
Huffman & Compressão & 0.161 & 176 \\
Huffman & Descompressão & 0.008 & 688 \\
Deflate & Compressão & 0.161 & 176 \\
Deflate & Descompressão & 0.008 & 688 \\
RLE & Compressão & 0.024 & 1232 \\
RLE & Descompressão & 0.005 & 680 \\
LZW & Compressão & 3.515 & 228 \\
LZW & Descompressão & 0.011 & 688 \\
\hline
\end{tabular}
\end{center}

No ambiente x86, os tempos de processamento foram significativamente menores em comparação com o ARMv6, indicando uma melhoria considerável de desempenho, especialmente para o algoritmo LZW.

\section{Discussão dos Resultados}

Os resultados dos testes demonstram claramente a influência do hardware no desempenho da ferramenta de compressão de dados. Ambos os ambientes apresentaram variações significativas nos tempos de compressão e descompressão, destacando a importância do hardware na eficiência dos algoritmos utilizados.

Além disso, os dados obtidos fornecem informações valiosas sobre os tempos de execução dos diferentes algoritmos em diferentes ambientes, contribuindo para a compreensão do desempenho da ferramenta em contextos variados.

    \chapter{Conclusões}
\label{cap:conclusoes}

Neste trabalho, apresentamos uma ferramenta de compressão de dados utilizando diferentes algoritmos e realizamos testes em ambientes distintos para avaliar seu desempenho. Com base nos resultados obtidos, as seguintes conclusões podem ser destacadas:

\section{Análise dos Resultados}

Os testes realizados demonstraram a influência significativa do hardware no desempenho da ferramenta. O ambiente x86 apresentou tempos de processamento consideravelmente menores em comparação ao processador ARMv6, evidenciando a importância do hardware na eficiência dos algoritmos de compressão.

\section{Considerações Finais}

Com base nos dados coletados, é possível inferir que a escolha do hardware é crucial para a otimização do desempenho da ferramenta. A análise dos resultados revelou diferenças marcantes nos tempos de compressão e descompressão entre os ambientes testados, indicando que investimentos em hardware mais robusto podem resultar em melhorias significativas no desempenho da ferramenta de compressão de dados.

\section{Trabalhos Futuros}

Como possíveis trabalhos futuros, sugerimos:
\begin{itemize}
    \item Explorar e implementar estratégias de otimização de algoritmos para melhorar o desempenho em ambientes de hardware menos robusto.
    \item Realizar testes em uma variedade maior de hardware para avaliar a escalabilidade e generalização do desempenho da ferramenta.
    \item Investigar a aplicação de novos algoritmos de compressão e comparar seu desempenho com os algoritmos atualmente implementados.
    \item Desenvolver versões específicas da ferramenta para diferentes tipos de hardware, visando a otimização e adaptação para cada ambiente específico.
\end{itemize}

Essas possibilidades de trabalhos futuros visam aprimorar a ferramenta, explorando diferentes estratégias e algoritmos para melhorar sua eficiência e aplicabilidade em uma variedade de ambientes de hardware.

    \phantompart

%  -----------------------------------------------------------------------
% |                      Inclui elementos pós-textuais                    |
%  -----------------------------------------------------------------------
    \postextual
    \bibliography{Bibliografia}
    \begin{apendicesenv}

\chapter{Exemplo de Apêndice}
\label{cap:apendice}

    Exemplo de Apêndice. Aqui, aproveitamos para descrever a regra para inserção de Figuras. As Figuras devem ser centralizadas, com legenda e fonte descritas alinhadas à esquerda, como pode ser observado pela Figura \ref{fig:exemplo}. O mesmo ocorre para os gráficos, como apresentado pelo Gráfico \ref{fig:exgrafico}.

    \begin{figure}[h]
        \caption{Exemplo de Imagem.}
        \begin{center}
            \includegraphics[width=0.6\linewidth]{Imagens/Tela.png}
        \end{center}
        \legend{\small{Fonte: \citeonline{SantanadaSilva2020}}}
        \label{fig:exemplo}
    \end{figure}

    \begin{grafico}[h]
        \caption{Exemplo de Gráfico.}
        \begin{center}
            \includegraphics[width=0.6\linewidth]{Imagens/grafico.png}
        \end{center}
        \label{fig:exgrafico}
    \end{grafico}

    Ainda, tabelas que possuem apenas informações textuais, devem ser nomeadas por quadros, como apresenta o Quadro \ref{qua:exemplo}.

    \begin{quadro}[ht]
        \caption{Exemplo de Quadro}
        \begin{tabular}{ll}
            \toprule
            Termo & Significado \\
            \midrule
            char  & Um caractere, ou valor inteiro não sinalizado com 8 bits \\
            int   & Valor numérico com 4 bytes de comprimento \\
            \bottomrule
            \bottomrule
        \end{tabular}
        \label{qua:exemplo}
    \end{quadro}

\end{apendicesenv}

    \begin{anexosenv}

\chapter{Exemplo de Anexo}
\label{cap:anexo}

    Este é um exemplo de anexo. Aqui aproveitamos para apresentar a construção de tabelas. As tabelas, assim como figuras, devem ser nomeadas e conter legendas alinhadas à esquerda, como apresenta a Tabela \ref{tab:exemplo}.

\begin{table}[htbp]
  \centering
  \caption{Exemplo de tabela}
    \begin{tabular}{rrrrrr}
    \toprule
          & \multicolumn{1}{l}{Altura} & \multicolumn{1}{l}{Prop.A} & \multicolumn{1}{l}{Prop.B} & \multicolumn{1}{l}{Prop.C} & \multicolumn{1}{l}{Prop.D} \\
    \midrule
    1     & 152.3115 & -19.6 & 30.88 & 33.12 & 0 \\
    2     & 161.3583 & 20.8959 & 30.49673 & 23.39696 & 10.32693 \\
    3     & 184.5315 & 23.89683 & 34.87646 & 26.75707 & 11.81002 \\
    4     & 175.0984 & 22.67524 & 33.0936 & 25.38927 & 11.2063 \\
    5     & 187.0856 & 24.22758 & 35.35918 & 27.12741 & 11.97348 \\
    6     & 164.3629 & 21.28499 & 31.06458 & 23.83262 & 10.51922 \\
    7     & 194.6153 & 25.20268 & 36.78229 & 28.21921 & 12.45538 \\
    8     & 163.868 & 21.22091 & 30.97105 & 23.76086 & 10.48755 \\
    9     & 163.5961 & 21.1857 & 30.91966 & 23.72144 & 10.47015 \\
    10    & 195.9685 & 25.37792 & 37.03805 & 28.41543 & 12.54198 \\
    11    & 194.2495 & 25.15531 & 36.71316 & 28.16618 & 12.43197 \\
    12    & 202.0585 & 26.16658 & 38.18906 & 29.29848 & 12.93174 \\
    13    & 183.856 & 23.80935 & 34.74879 & 26.65912 & 11.76679 \\
    14    & 168.8063 & 21.86041 & 31.90439 & 24.47691 & 10.8036 \\
    15    & 180.949 & 23.4329 & 34.19936 & 26.23761 & 11.58074 \\
    16    & 169.6921 & 21.97512 & 32.0718 & 24.60535 & 10.86029 \\
    17    & 164.2326 & 21.26813 & 31.03997 & 23.81373 & 10.51089 \\
    18    & 203.7827 & 26.38986 & 38.51493 & 29.54849 & 13.04209 \\
    19    & 160.2441 & 20.75161 & 30.28614 & 23.2354 & 10.25562 \\
    20    & 164.6612 & 21.32363 & 31.12097 & 23.87588 & 10.53832 \\
    21    & 159.3832 & 20.64012 & 30.12342 & 23.11056 & 10.20052 \\
    22    & 141.8475 & 18.36925 & 26.80917 & 20.56789 & 9.078239 \\
    23    & 168.5719 & 21.83006 & 31.86009 & 24.44292 & 10.7886 \\
    24    & 183.5559 & 23.77049 & 34.69207 & 26.61561 & 11.74758 \\
    25    & 164.6031 & 21.3161 & 31.10999 & 23.86745 & 10.5346 \\
    26    & 185.039 & 23.96255 & 34.97237 & 26.83065 & 11.84249 \\
    27    & 177.4706 & 22.98245 & 33.54195 & 25.73324 & 11.35812 \\
    28    & 216.0425 & 27.97751 & 40.83204 & 31.32616 & 13.82672 \\
    29    & 195.0084 & 25.25359 & 36.85658 & 28.27622 & 12.48054 \\
    30    & 162.5406 & 21.04901 & 30.72018 & 23.56839 & 10.4026 \\
    \bottomrule
    \bottomrule
    \end{tabular}%
    \legend{\small{Os autores podem descrever Fonte ou outras informações sobre a Tabela}}
  \label{tab:exemplo}%
\end{table}%

    
\end{anexosenv}

\end{document}
