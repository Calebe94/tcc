\chapter{Introdução}
\label{cap:introducao}

A Internet das Coisas (IoT) tem revolucionado a forma como interagimos com o ambiente ao nosso redor, conectando dispositivos, sensores e sistemas a uma rede global, permitindo a coleta e troca de informações em tempo real. Esta evolução tecnológica tem encontrado aplicações em diversos setores, desde a automação residencial até a otimização de processos industriais e sistemas de saúde avançados. No entanto, à medida que a IoT se expande, torna-se cada vez mais evidente a necessidade de enfrentar os desafios associados ao armazenamento e transmissão de dados em dispositivos com recursos limitados.

\section{Problema}
\label{sec:problema}

O advento da Internet das Coisas (IoT) revolucionou a forma como interagimos com o mundo ao nosso redor, conectando dispositivos, sensores e sistemas a uma rede global. No entanto, esse crescimento exponencial de dispositivos IoT trouxe consigo desafios significativos relacionados ao armazenamento e transmissão eficiente de dados. Muitos dispositivos IoT são caracterizados por recursos limitados, incluindo capacidade de armazenamento, capacidade de processamento e largura de banda de transmissão. Este cenário levanta a seguinte questão fundamental: como gerenciar eficazmente os dados gerados por dispositivos IoT em um ambiente de recursos limitados?

\section{Justificativa}
\label{sec:justificativa}

A justificativa para a realização deste trabalho reside na necessidade premente de abordar os desafios inerentes à IoT, particularmente no que diz respeito ao gerenciamento eficiente de dados em dispositivos com recursos limitados. Dispositivos IoT são amplamente utilizados em diversas aplicações, incluindo saúde, agricultura, manufatura e cidades inteligentes, e o volume de dados gerados por esses dispositivos continua a crescer exponencialmente. A compressão de dados surge como uma estratégia crucial para otimizar a utilização de recursos, economizar energia e melhorar a eficiência na transmissão de dados em redes com baixa taxa de transferência. Portanto, este trabalho se justifica pela necessidade de explorar e avaliar as técnicas de compressão de dados mais apropriadas para dispositivos IoT com recursos limitados, contribuindo para a resolução deste desafio tecnológico.

\section{Objetivo Geral}
\label{sec:ObjGeral}

O objetivo geral deste trabalho é investigar e analisar as técnicas de compressão de dados mais adequadas para dispositivos IoT com recursos limitados de armazenamento e transmissão. Isso envolve a compreensão das diferentes abordagens de compressão disponíveis, suas vantagens, desvantagens e áreas de aplicação ideais em contextos de IoT. Além disso, o objetivo geral inclui a realização de estudos de caso práticos para avaliar a eficácia dessas técnicas em cenários reais de implementação de dispositivos IoT.

\section{Objetivos Específicos}
\label{sec:ObjEspecificos}

Para atingir o objetivo geral, os seguintes objetivos específicos foram estabelecidos:

\begin{itemize}
\item Realizar uma revisão abrangente da literatura existente sobre técnicas de compressão de dados, com foco nas estratégias desenvolvidas para dispositivos IoT com recursos limitados.
\item Comparar as técnicas de compressão de dados sem perdas e com perdas, identificando suas características, vantagens e limitações.
\item Implementar um estudo de caso prático que envolva a coleta de dados por dispositivos IoT em um ambiente controlado.
\item Aplicar diversas técnicas de compressão de dados aos dados coletados e analisar os resultados em termos de economia de recursos, qualidade dos dados e eficiência da transmissão.
\end{itemize}
