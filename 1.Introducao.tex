%  -----------------------------------------------------------------------
% |                        Capítulo de Introdução                         |
%  -----------------------------------------------------------------------
\chapter{Introdução}
\label{cap:introducao}

Este documento apresenta a organização mínima e obrigatória de capítulos e seções dos elementos textuais para composição do documento de Trabalho de Conclusão de Curso (TCC) de Engenharia da Computação, seguindo diretrizes da NBR 14724:2011 \cite{ABNT2011}. 
\nomenclature{TCC}{Trabalho de Conclusão de Curso}

Em concordância com a Instrução Normativa (IN) de Trabalho de Conclusão de Curso, do curso de Engenharia da Computação, serão descritos os requisitos de organização para os trabalhos do tipo “Protótipo” e “Projeto de Pesquisa”.
\nomenclature{IN}{Instrução Normativa}

Este documento não exclui a \underline{obrigatoriedade} do cumprimento de regras de formatação definidas pela Biblioteca da \citeonline{UniversidadePositivoUP2019}, bem como da inserção de elementos pré e pós textuais. Tais normas podem ser acessadas em \url{www.up.edu.br/biblioteca/abnt---normas}

Ao longo do Capítulo de Introdução os autores deverão versar sobre o problema observado pelo trabalho, sua justificativa, Objetivo Geral e Objetivos Específicos. Tais itens deverão ser apresentados, \textbf{\underline{obrigatoriamente}}, em suas seções específicas.

\section{Problema}
\label{sec:problema}

O que você deseja resolver? Procure contextualizar o problema através de informações científicas e atuais, apresentando um cenário abrangente.

\section{Justificativa}
\label{sec:justificativa}

Por que o problema apresentado merece seu olhar? Como, e a que nível, seu projeto poderá auxiliar a sanar o problema? 

\section{Objetivo Geral}
\label{sec:ObjGeral}

Qual é o objetivo geral do seu projeto? Descreva em uma frase, de forma clara e completa o que deseja alcançar ao final da execução deste trabalho. A menos que seja altamente necessário, não nomeie dispositivos, algoritmos e \textit{frameworks}. Leve em consideração que o objetivo foi definido na proposta do trabalho.

\section{Objetivos Específicos}
\label{sec:ObjEspecificos}

Quais são os objetivos intermediários que serão alcançados para que o objetivo geral seja cumprido? Observe que estes objetivos podem ser considerados passos para cumprir o objetivo. Não há limites para a quantidade de objetivos específicos, porém, seja coerente.

Os objetivos específicos são compostos por apenas um verbo. Cuidado; estudar, conhecer, pesquisar, entre outros verbos, não são considerados componentes aceitáveis para composição dos objetivos específicos.

\begin{itemize}
    \item Objetivo específico 1;
    \item Objetivo específico 2;
    \item ...
\end{itemize}

